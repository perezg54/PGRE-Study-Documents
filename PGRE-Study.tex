% ****** Start of file apssamp.tex ******
%
%   This file is part of the APS files in the REVTeX 4.1 distribution.
%   Version 4.1r of REVTeX, August 2010
%
%   Copyright (c) 2009, 2010 The American Physical Society.
%
%   See the REVTeX 4 README file for restrictions and more information.
%
% TeX'ing this file requires that you have AMS-LaTeX 2.0 installed
% as well as the rest of the prerequisites for REVTeX 4.1
%
% See the REVTeX 4 README file
% It also requires running BibTeX. The commands are as follows:
%
%  1)  latex apssamp.tex
%  2)  bibtex apssamp
%  3)  latex apssamp.tex
%  4)  latex apssamp.tex
%
\documentclass[%
 reprint,
superscriptaddress,
%groupedaddress,
%unsortedaddress,
%runinaddress,
%frontmatterverbose, 
%preprint,
%showpacs,preprintnumbers,
%nofootinbib,
%nobibnotes,
%bibnotes,
 amsmath,amssymb,
 aps,
prc,
%prb,
%rmp,
%prstab,
%prstper,
%floatfix,
]{revtex4-1}

\usepackage{graphicx}% Include figure files
\usepackage{tabularx}
\newcolumntype{C}{>{\centering\arraybackslash}X}
\newcolumntype{L}{>{\raggedright\arraybackslash}X}%
\newcolumntype{R}{>{\raggedleft\arraybackslash}X}%
\usepackage{dcolumn}% Align table columns on decimal point
\usepackage{bm}% bold math
\usepackage{hyperref}% add hypertext capabilities
\usepackage[mathlines]{lineno}% Enable numbering of text and display math
%\linenumbers\relax % Commence numbering lines
\usepackage{circuitikz}
\usepackage{tikz}
\usepackage{xcolor}
\hypersetup{
    colorlinks,
    linkcolor={red!50!black},
    citecolor={blue!50!black},
    urlcolor={blue!80!black}
}
\usepackage{graphicx}
%\usepackage[showframe,%Uncomment any one of the following lines to test 
%%scale=0.7, marginratio={1:1, 2:3}, ignoreall,% default settings
%%text={7in,10in},centering,
%%margin=1.5in,
%%total={6.5in,8.75in}, top=1.2in, left=0.9in, includefoot,
%%height=10in,a5paper,hmargin={3cm,0.8in},
%]{geometry}
\usepackage{amsmath}
\usepackage{amssymb}
\begin{document}

\preprint{APS/123-QED}

\title{PGRE: Studying for Death?}% Force line breaks with \\

\author{Gray Ezequiel G.B. Perez}
\author{Schlack-Dog}
\affiliation{%
 Reed College, 3203 SE Woodstock Blvd, Portland, OR 97202, USA
}%

\begin{abstract}
Attempting to avoid death by making a big study guide!
\end{abstract}
\maketitle
\tableofcontents
\section{Variable definitions}
% consistent variable definitions? or should this just be a list of standard ones?
\begin{enumerate}
\item[] $a\rightarrow$ acceleration

\item[] $c\rightarrow$ speed of light:\\
standard  $\approx 3*10^8\dfrac{m}{s}$
\end{enumerate}

\section{Classical mechanics (20\%)}
% ETS description of the subject in brief
Such as kinematics, Newton's laws, work and energy, oscillatory motion, rotational motion about a fixed axis, dynamics of systems of particles, central forces and celestial mechanics, three-dimensional particle dynamics, Lagrangian and Hamiltonian formalism, noninertial reference frames, elementary topics in fluid dynamics.

\section{Electromagnetism (18\%)}
% ETS description of the subject in brief
Such as electrostatics, currents and DC circuits, magnetic fields in free space, Lorentz force, induction, Maxwell's equations and their applications, electromagnetic waves, AC circuits, magnetic and electric fields in matter.

\subsection{Maxwell's Equations}

\begin{itemize}
	\item \textbf{Speed of light:} The definition (in vacuum) can be found by taking the curl of Maxwell's equations and looking for something in the form of the wave equation. It is
	\begin{equation}
	    c=\frac{1}{\sqrt{\mu_0\epsilon_0}}.
	\end{equation}
	For light in matter, replace $\mu_0$ and $\epsilon_0$ with $\mu$ and $\epsilon$, respectively.
\end{itemize}

\subsection{Magnetic and Electric Fields in Matter}

\begin{itemize}
	\item \textbf{Permittivity:} In matter, the permittivity is given by
	\begin{equation}
		\epsilon = \kappa_E \epsilon_0,
	\end{equation}
	where $\epsilon$ is the absolute permittivity, $\kappa_E$ is the dielectric constant, and $\epsilon_0$ is vacuum permittivity. 
	
	If the applied field is not constant, then $\epsilon$ becomes frequency-dependent because the material's polarization does not change instantly. In this case
	\begin{equation}
		\epsilon (\omega) = \kappa_E\epsilon_0 - i \frac{\sigma}{\omega},
	\end{equation}
	where $\sigma$ is the conductivity of the material and $\omega$ is the frequency of the applied field.
	
	Note that electric susceptibility is related to the dielectric constant with the simple relation
	\begin{equation}
		\chi_E = \kappa_E - 1.
	\end{equation} 
	
	\item \textbf{Permeability:} The permeability is given by
	\begin{equation}
		\mu = \kappa_B \mu_0,
	\end{equation}
	where $\mu$ is the absolute permeability, $\kappa_B$ is a constant, and $\mu_0$ is vacuum permeability. $\kappa_B$ is related to the magnetic susceptibility by
	\begin{equation}
		\chi_B = \kappa_B - 1.
	\end{equation}
	
	Similarly to electric fields in matter, $\mu$ does have a frequency dependence. However, this dependence is negligible in non-magnetic materials.
	
	Fun fact: a material with $\mu < \mu_0$ is \textbf{diamagnetic}, and a material with $\mu > \mu_0$ is \textbf{paramagnetic}.
\end{itemize}


\section{Quantum Mechanics (12\%)}
% ETS description of the subject in brief
Such as fundamental concepts, solutions of the Schrödinger equation (including square wells, harmonic oscillators, and hydrogenic atoms), spin, angular momentum, wave function symmetry, elementary perturbation theory.

\section{Thermodynamics and Statistical Mechanics (10\%)}
% ETS description of the subject in brief
Such as the laws of thermodynamics, thermodynamic processes, equations of state, ideal gases, kinetic theory, ensembles, statistical concepts and calculation of thermodynamic quantities, thermal expansion and heat transfer.

\section{Atomic Physics (10\%)}
% ETS description of the subject in brief
Such as properties of electrons, Bohr model, energy quantization, atomic structure, atomic spectra, selection rules, black-body radiation, x-rays, atoms in electric and magnetic fields.

\section{Optics and Wave Phenomena (9\%)}
% ETS description of the subject in brief
Such as wave properties, superposition, interference, diffraction, geometrical optics, polarization, Doppler effect.

\subsection{Doppler Effect}
The general form of the (classical) Doppler effect is
\begin{equation}
	f=\frac{c \pm v_\text{r}}{c \pm v_\text{s}} f_0,
\end{equation}
where $f$ is the observed frequency, $f_0$ is the source frequency, $c$ is the velocity of the waves in medium, $v_\text{r}$ is the velocity of the receiver relative to the medium, and $v_\text{s}$ is the velocity of the source relative to the medium. $v_\text{r}$ is positive if the receiver is moving towards the source and negative if it is moving away from the source. $v_\text{s}$ is positive if the source is moving away from the receiver and negative if it is towards the receiver.

In the relativistic case, the velocities above should be added using the velocity addition formula. This gives 
\begin{equation}
	f = \sqrt{\frac{1 - v/c}{1 + v/c}} f_0,
\end{equation}
where $v$ is the velocity of the source relative to the observer. $v$ is positive if the source and observer are receding from each other, and negative if the source and observer are approaching each other.

\section{Specialized Topics (9\%)}
% ETS description of the subject in brief
Nuclear and Particle physics (e.g., nuclear properties, radioactive decay, fission and fusion, reactions, fundamental properties of elementary particles), Condensed Matter (e.g., crystal structure, x-ray diffraction, thermal properties, electron theory of metals, semiconductors, superconductors), Miscellaneous (e.g., astrophysics, mathematical methods, computer applications)

\section{Special Relativity (6\%)}
% ETS description of the subject in brief
Such as introductory concepts, time dilation, length contraction, simultaneity, energy and momentum, four-vectors and Lorentz transformation, velocity addition.

\section{Laboratory Methods (6\%)}
% ETS description of the subject in brief
Such as data and error analysis, electronics, instrumentation, radiation detection, counting statistics, interaction of charged particles with matter, lasers and optical interferometers, dimensional analysis, fundamental applications of probability and statistics.

\subsection{Error Analysis}
\begin{itemize}
	\item \textbf{Counting error:} the error in counting a sample of size $N$ is given by
	\begin{equation}
		\Delta N = \sqrt{N}.
	\end{equation}
\end{itemize}

\end{document}