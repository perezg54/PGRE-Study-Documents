% ****** Start of file apssamp.tex ******
%
%   This file is part of the APS files in the REVTeX 4.1 distribution.
%   Version 4.1r of REVTeX, August 2010
%
%   Copyright (c) 2009, 2010 The American Physical Society.
%
%   See the REVTeX 4 README file for restrictions and more information.
%
% TeX'ing this file requires that you have AMS-LaTeX 2.0 installed
% as well as the rest of the prerequisites for REVTeX 4.1
%
% See the REVTeX 4 README file
% It also requires running BibTeX. The commands are as follows:
%
%  1)  latex apssamp.tex
%  2)  bibtex apssamp
%  3)  latex apssamp.tex
%  4)  latex apssamp.tex
%
\documentclass[%
 reprint,
superscriptaddress,
%groupedaddress,
%unsortedaddress,
%runinaddress,
%frontmatterverbose, 
%preprint,
%showpacs,preprintnumbers,
%nofootinbib,
%nobibnotes,
%bibnotes,
 amsmath,amssymb,
 aps,
prc,
%prb,
%rmp,
%prstab,
%prstper,
%floatfix,
]{revtex4-1}

\usepackage{graphicx}% Include figure files
\usepackage{tabularx}
\newcolumntype{C}{>{\centering\arraybackslash}X}
\newcolumntype{L}{>{\raggedright\arraybackslash}X}%
\newcolumntype{R}{>{\raggedleft\arraybackslash}X}%
\usepackage{dcolumn}% Align table columns on decimal point
\usepackage{bm}% bold math
\usepackage{hyperref}% add hypertext capabilities
\usepackage[mathlines]{lineno}% Enable numbering of text and display math
%\linenumbers\relax % Commence numbering lines
\usepackage{circuitikz}
\usepackage{tikz}
\usepackage{xcolor}
\hypersetup{
    colorlinks,
    linkcolor={red!50!black},
    citecolor={blue!50!black},
    urlcolor={blue!80!black}
}
\usepackage{graphicx}
%\usepackage[showframe,%Uncomment any one of the following lines to test 
%%scale=0.7, marginratio={1:1, 2:3}, ignoreall,% default settings
%%text={7in,10in},centering,
%%margin=1.5in,
%%total={6.5in,8.75in}, top=1.2in, left=0.9in, includefoot,
%%height=10in,a5paper,hmargin={3cm,0.8in},
%]{geometry}
\usepackage{amsmath}
\usepackage{amssymb}
\begin{document}

\preprint{APS/123-QED}

\title{PGRE: Studying for Death?}% Force line breaks with \\

\author{Gray Ezequiel G.B. Perez}
\author{Schlack-Doggo}
\affiliation{%
 Reed College, 3203 SE Woodstock Blvd, Portland, OR 97202, USA
}%

\begin{abstract}
Attempting to avoid death by making a big study guide!
\end{abstract}
\maketitle
\tableofcontents
\section{Variable definitions}
% consistent variable definitions? or should this just be a list of standard ones?
\begin{enumerate}
\item[] $a\rightarrow$ acceleration

\item[] $c\rightarrow$ speed of light:\\
standard  $\approx 3*10^8\dfrac{m}{s}$

\end{enumerate}

\section{Classical mechanics (20\%)}
% ETS description of the subject in brief
Such as kinematics, Newton's laws, work and energy, oscillatory motion, rotational motion about a fixed axis, dynamics of systems of particles, central forces and celestial mechanics, three-dimensional particle dynamics, Lagrangian and Hamiltonian formalism, noninertial reference frames, elementary topics in fluid dynamics.

\section{Electromagnetism (18\%)}
% ETS description of the subject in brief
Such as electrostatics, currents and DC circuits, magnetic fields in free space, Lorentz force, induction, Maxwell's equations and their applications, 
electromagnetic waves, AC circuits, magnetic and electric fields in matter.
\subsection{Electric Displacement Field}
On the shallow level the \textbf{Electric Displacement} is the equivalent of an electric field in a media. Specifically it is defined as:
$$\vec{D}\equiv \epsilon_0 \vec{E}+\vec{P}, $$
where $\vec{E}$ is the \textbf{electric field}, $\epsilon_0$ is the \textbf{vacuum permittivity}, and $\vec{P}$ is the \textbf{polarization}. Importantly 
the electric displacement appears in the in-media version of Maxwell's equations. Specifically,
$$\nabla \cdot D = \rho_f,$$
where $\rho_f$ is the \textbf{free charge density}, that which is not the \textbf{bound charge density}. We can prove this by first noting that $\rho_f$ and 
$\rho_b$ are two complementary subsets of the total charge density $\rho$ i.e.
$$\rho=\rho_f+ \rho_b.$$
next we not that the bound charge density is given by:
$$-\nabla\cdot\vec{P}=\rho_b.  $$
From this we get,
\begin{align*}
\nabla \cdot \vec{E} &=\dfrac{\rho}{\epsilon_0}\\
&=\dfrac{1}{\epsilon_0}\left(\rho_f-\nabla\cdot\vec{P}\right)\\
\nabla\cdot\left(\epsilon_0\vec{E}+\vec{P}\right)=\nablda\cdot\vec{D}&=\rho_f.
\end{align*}
if we limit ourselves to linear dielectrics according to the big G (or to wikipedia a linear, homogenous, isotropic dielectric with instantaneous response 
to changes in the electric field) we can define the polarization as:
$$\vec{P}=\epsilon_0\chi\vec{E}.$$
where $\chi$ is the \textbf{electric susceptibility} of the material. Now we define for our convenience and sorrow
$$\epsilon_r\equiv 1+\chi,$$
the \textbf{relative permittivity} of the material and
$$\epsilon=\epsilon_0\epsilon_r,$$
the \textbf{permittivity} of the material such that
$$\vec{D}=\epsilon\vec{E}.$$


\section{Quantum Mechanics (12\%)}
% ETS description of the subject in brief
Such as fundamental concepts, solutions of the Schrödinger equation (including square wells, harmonic oscillators, and hydrogenic atoms), spin, angular 
momentum, wave function symmetry, elementary perturbation theory.

\section{Thermodynamics and Statistical Mechanics (10\%)}
% ETS description of the subject in brief
Such as the laws of thermodynamics, thermodynamic processes, equations of state, ideal gases, kinetic theory, ensembles, statistical concepts and calculation of thermodynamic quantities, thermal expansion and heat transfer.
\subsection{Thermodynamic Laws}
\subsubsection{1st Law}

\section{Atomic Physics (10\%)}
% ETS description of the subject in brief
Such as properties of electrons, Bohr model, energy quantization, atomic structure, atomic spectra, selection rules, black-body radiation, x-rays, atoms in electric and magnetic fields.

\section{Optics and Wave Phenomena (9\%)}
% ETS description of the subject in brief
Such as wave properties, superposition, interference, diffraction, geometrical optics, polarization, Doppler effect.

\section{Specialized Topics (9\%)}
% ETS description of the subject in brief
Nuclear and Particle physics (e.g., nuclear properties, radioactive decay, fission and fusion, reactions, fundamental properties of elementary particles), Condensed Matter (e.g., crystal structure, x-ray diffraction, thermal properties, electron theory of metals, semiconductors, superconductors), Miscellaneous (e.g., astrophysics, mathematical methods, computer applications)

\section{Special Relativity (6\%)}
% ETS description of the subject in brief
Such as introductory concepts, time dilation, length contraction, simultaneity, energy and momentum, four-vectors and Lorentz transformation, velocity addition.

\section{Laboratory Methods (6\%)}
% ETS description of the subject in brief
Such as data and error analysis, electronics, instrumentation, radiation detection, counting statistics, interaction of charged particles with matter, lasers and optical interferometers, dimensional analysis, fundamental applications of probability and statistics.

\end{document}