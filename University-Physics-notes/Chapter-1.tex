% ****** Start of file apssamp.tex ******
%
%   This file is part of the APS files in the REVTeX 4.1 distribution.
%   Version 4.1r of REVTeX, August 2010
%
%   Copyright (c) 2009, 2010 The American Physical Society.
%
%   See the REVTeX 4 README file for restrictions and more information.
%
% TeX'ing this file requires that you have AMS-LaTeX 2.0 installed
% as well as the rest of the prerequisites for REVTeX 4.1
%
% See the REVTeX 4 README file
% It also requires running BibTeX. The commands are as follows:
%
%  1)  latex apssamp.tex
%  2)  bibtex apssamp
%  3)  latex apssamp.tex
%  4)  latex apssamp.tex
%
\documentclass[%
 reprint,
superscriptaddress,
%groupedaddress,
%unsortedaddress,
%runinaddress,
%frontmatterverbose, 
%preprint,
%showpacs,preprintnumbers,
%nofootinbib,
%nobibnotes,
%bibnotes,
 amsmath,amssymb,
 aps,
prc,
%prb,
%rmp,
%prstab,
%prstper,
%floatfix,
]{revtex4-1}

\usepackage{graphicx}% Include figure files
\usepackage{tabularx}
\newcolumntype{C}{>{\centering\arraybackslash}X}
\newcolumntype{L}{>{\raggedright\arraybackslash}X}%
\newcolumntype{R}{>{\raggedleft\arraybackslash}X}%
\usepackage{dcolumn}% Align table columns on decimal point
\usepackage{bm}% bold math
\usepackage{hyperref}% add hypertext capabilities
\usepackage[mathlines]{lineno}% Enable numbering of text and display math
%\linenumbers\relax % Commence numbering lines
\usepackage{circuitikz}
\usepackage{tikz}
\usepackage{xcolor}
\hypersetup{
    colorlinks,
    linkcolor={red!50!black},
    citecolor={blue!50!black},
    urlcolor={blue!80!black}
}
\usepackage{graphicx}
%\usepackage[showframe,%Uncomment any one of the following lines to test 
%%scale=0.7, marginratio={1:1, 2:3}, ignoreall,% default settings
%%text={7in,10in},centering,
%%margin=1.5in,
%%total={6.5in,8.75in}, top=1.2in, left=0.9in, includefoot,
%%height=10in,a5paper,hmargin={3cm,0.8in},
%]{geometry}
\usepackage{amsmath}
\usepackage{amssymb}
\begin{document}

\preprint{APS/123-QED}

\title{University Physics with modern Physics}% Force line breaks with \\

\author{Gray Ezequiel G.B. Perez}

\affiliation{%
 Reed College, 3203 SE Woodstock Blvd, Portland, OR 97202, USA
}%

\begin{abstract}
Attempting to avoid death by studying too much!
\end{abstract}
\maketitle
\tableofcontents
\section{Units, Physical Quantities, and Vectors}
\subsection{Discussion Questions}
\subsubsection{}
It only takes one inconsistent result to disprove a theory. Of course having multiple adds certainty, but ultimately one is enough if one can be reasonably 
sure the results it correct. Proving a theory is impossible, one can only increase our confidence by many experiments that are consistent.

\subsubsection{}
you could convert to the same units and get $0.12$ and call that a difficulty rating after normalizing it to some upper difficulty value.

\subsubsection{}
This is not possible because a function such as tangent cannot take units in as an argument, we could take the tangent of $5$ and interpret this as giving
units of meters out.

\subsubsection{}
he probably meant something like a $\text{yard}^2$ with some standard depth, say an inch

\subsubsection{}
my height is 170.18 cm and weight is 75 kilograms.

\subsubsection{}
The idea of the kilogram and its value is not changing, only objects we use to keep track of it. At that rate it would take 1,000,000 years to change one 
of them by a gram, but for accurate purposes this could perhaps still be an issue. ultimately it's not a huge concern.

\subsubsection{}
one could use rotation of the earth about the sun, this would be kinda poor though. we could use basically whatever we wanted really.

\subsubsection{}
I could fold the paper repeatedly, and well, then measure the new total thickness and just assume even thickness per layer!

\subsubsection{}
the natural number $e$, and $\tau$, then any ratio really

\subsubsection{}
standard units of volume are meters cubed. That answer would have the wrong units!

\subsubsection{}
\begin{tabular}{|c|c|c|}
\hline
name & Accurate & Precise\\
\hline
Joe & yes & no\\
Moe & no & yes\\
Flo & yes & yes\\
\hline
\end{tabular}

\subsubsection{}
First is $500m$ then is $0m$ since she has returned to her initial position.

\subsubsection{}

\subsection{Exercises}
\subsubsection{}
\begin{enumerate}
\item[a)] There are 5280 feet in a mile so
\begin{equation*}
1\text{mile}\dfrac{5280\text{feet}}{1\text{mile}}\dfrac{12\text{inches}}{1\text{foot}}\dfrac{2.54\text{cm}}{1\text{inch}}\dfrac{1\text{m}}{100\text{cm}}\dfrac{1\text{km}}{1000\text{m}}=1.609\text{km}
\end{equation*} 
\item[b)]
\begin{equation}
1km\left(\dfrac{1000m}{1km}\right)\left(\dfrac{100cm}{1m}\right)\left(\dfrac{1in}{2.54cm}\right)\left(\dfrac{1ft}{12in}\right)=3,281ft
\end{equation}
\end{enumerate}
\subsubsection{}
\begin{equation}
0.473L\left(\dfrac{1000cm^3}{1L}\right)\left(\dfrac{1in}{2.54cm}\right)^3=28.9in^3
\end{equation}
\subsubsection{}
\end{document}